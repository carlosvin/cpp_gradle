%% LyX 2.0.1 created this file.  For more info, see http://www.lyx.org/.
%% Do not edit unless you really know what you are doing.
\documentclass[spanish]{article}
\usepackage{lmodern}
\usepackage[T1]{fontenc}
\usepackage[utf8]{luainputenc}
\usepackage{listings}
\usepackage{babel}
\addto\shorthandsspanish{\spanishdeactivate{~<>}}

\usepackage{float}
\usepackage[unicode=true,pdfusetitle,
 bookmarks=true,bookmarksnumbered=false,bookmarksopen=false,
 breaklinks=false,pdfborder={0 0 0},backref=false,colorlinks=false]
 {hyperref}

\makeatletter

%%%%%%%%%%%%%%%%%%%%%%%%%%%%%% LyX specific LaTeX commands.
\floatstyle{ruled}
\newfloat{algorithm}{tbp}{loa}
\providecommand{\algorithmname}{Algoritmo}
\floatname{algorithm}{\protect\algorithmname}

\makeatother

\begin{document}

\title{Construir un proyecto C++ con Gradle}

\maketitle

\section{Introducción}

La construcción, gestión de dependencias y distribución de mis proyectos
es algo que cada vez me preocupa más, me gustaría encontrar una herramienta
que unificara este proceso y fuese independiente del lenguaje, lo
más parecido con lo que he trabajado ha sido \href{http://www.scons.org/}{SCons},
\href{http://www.gnu.org/software/automake/manual/html_node/Autotools-Introduction.html\#Autotools-Introduction}{Autotools},
\href{http://ant.apache.org/}{Ant}, \href{http://maven.apache.org/}{Maven}
y últimamente \href{http://www.gradle.org/}{Gradle}. 

Llevo un tiempo haciendo algunas cosas con \href{http://www.gradle.org/}{Gradle},
pero siempre centrado en proyectos Java utilizándolo como reemplazo
a Maven, porque que es más sencillo de usar y más rápido. También
lo he utilizado \href{http://developer.android.com/sdk/installing/studio-build.html}{en projectos Android}
y he sufrido la pareja \href{http://developer.android.com/sdk/installing/studio-build.html}{Android Studio + Gradle}
en sus primeros matrimonios (porque yo he querido), actualmente está
todo mucho más documentado y funciona muy bien. 

Antes de nada hay que decir que la construcción de proyectos C/C++
y Objective-C con \href{http://www.gradle.org/}{Gradle} se encuentra
en fase de incubación, aunque ya permite hacer algunas tareas avanzadas
como:
\begin{itemize}
\item Generación de múltiples artefactos dentro del mismo proyecto, esto
es varias librerías o ejecutables.
\item Gestionar las dependencias entre estos artefactos (sin versiones).
\item Generar distintos ``sabores'' de aplicaciones, por ejemplo: podremos
generar una aplicación ``Community'' y otra con más características
habilitadas llamada ``Enterprise''.
\item Permite generar binarios para distintas plataformas, esto depende
de las cadenas de herramientas (\href{http://es.wikipedia.org/wiki/Cadena_de_herramientas}{Toolchains})
que tengamos instaladas en la máquina de compilación. 
\end{itemize}
Como decía todavía tiene limitaciones, aunque están trabajando en
ello y \href{http://www.gradleware.com/resources/cpp/}{si consiguen lo que tienen planeado }dejaré
\href{http://www.gnu.org/software/automake/manual/html_node/Autotools-Introduction.html\#Autotools-Introduction}{Autotools}
(me arrepentiré de haber dicho esto).


\section{Un caso práctico}

Básicamente he sacado todo el ejemplo de \href{http://www.gradle.org/docs/current/userguide/nativeBinaries.html}{aquí}
y lo he adaptado a un caso en el que hay varias plataformas y quiero
generar dos versiones distintas de mi software ``Community'' y ``Enterprise''. 

La aplicación es un ejecutable y una librería dinámica. El ejecutable
hace uso de esta librería. Ya está, solo quiero mostrar lo que nos
permite hacer \href{http://www.gradle.org/}{Gradle}.

También nos permitirá generar una versión para distribuir y otra para
depurar. 

Todo el código se encuentra en \href{https://github.com/carlosvin/cpp_gradle}{https://github.com/carlosvin/cpp\_{}gradle}.


\subsection{Estructura del proyecto}

Podemos crear la estructura que queramos, pero resulta más fácil seguir
la que espera Gradle, para no tener que especificar donde está el
código fuente. Esta es la estructura del proyecto:
\begin{description}
\item [{gradle\_cpp}]~

\begin{description}
\item [{build.gradle}] Fichero donde se configura el proyecto Gradle, el
equivalente al build.xml de Ant, al Makefile de C/C++ o al pom.xml
de Maven.
\item [{src}] Carpeta donde va todo el código fuente

\begin{description}
\item [{hello}] Carpeta que contiene el módulo que va a ser la librería
hello. 

\begin{description}
\item [{cpp}] Carpeta donde van los fuentes C++.

\begin{description}
\item [{Hello.cpp}]~
\end{description}
\item [{headers}] Carpeta donde val los ficheros de cabeceras. 

\begin{description}
\item [{Hello.h}]~
\item [{Msg.h}]~
\end{description}
\end{description}
\item [{main}] Carpeta que contiene el módulo que será el ejecutable que
utilice la librería hello.

\begin{description}
\item [{cpp}] Carpeta donde van los fuentes C++.

\begin{description}
\item [{main.cpp}]~
\end{description}
\end{description}
\end{description}
\item [{build}] Carpeta que crea Gradle automáticamente donde deja todos
los resultados sus ejecuciones, en ella encontraremos informes de
resultados de pruebas, binarios compilados, paquetes para distribuir,
etc. 
\end{description}
\end{description}

\subsection{La aplicación C++}

Va a consistir en un ejecutable que hará uso de la funcionalidad de
la librería 'hello'. 

\begin{algorithm}[H]
\begin{lstlisting}[language={C++},tabsize=4]
#include "Hello.h"
int main(int argc, char ** argv) 
{ 	
	Hello hello ("Pepito");
	hello.sayHello(10);
	return 0; 
}
\end{lstlisting}


\caption{main.cpp}


\end{algorithm}


Esta librería permite saludar $n$ veces a una persona especificada
en su constructor. 

\begin{algorithm}[H]
\begin{lstlisting}[language={C++},tabsize=4]
class Hello  
{
	private:
		const char * who;
	public:
		Hello(const char * who);
		void sayHello(unsigned n = 1);
};
\end{lstlisting}


\caption{Hello.h}
\end{algorithm}



\subsection{Construyendo con Gradle}


\subsubsection{Caso básico}

Lo único que necesitamos para construir nuestra aplicación con Gradle
es: tener Gradle%
\footnote{Realmente no es necesario tener instalado Gradle, si utilizamos el
wrapper, pero esto no lo vamos a tratar hoy,\href{http://www.gradle.org/docs/current/userguide/nativeBinaries.html}{ si queréis más información}.%
} y el fichero build.gradle.

\begin{algorithm}[H]
\begin{lstlisting}
apply plugin: 'cpp'

libraries {  	
	hello {} 
}
executables {     
	main {
		binaries.all {
			lib libraries.hello.shared         
		}
    }
}
\end{lstlisting}
\caption{build.gradle}
\end{algorithm}


Con este fichero tan simple, conseguiremos compilar e instalar nuestra
aplicación, en modo Debug para la plataforma donde estamos ejecutando
gradle, en mi caso es Linux X64. 

Si ejecutamos desde la raíz de nuestro proyecto gradle task, podremos
ver todas las tareas que podemos hacer. 

En nuestro caso, solo queremos nuestra aplicación compilada y lista
para funcionar, así que ejecutaremos: gradle installMainExecutable.

Una vez que ha terminado, podemos ejecutar el programa llamando al
script %
\footnote{ .bat en Windows y .sh en Linux%
}.

\begin{algorithm}[H]
\begin{lstlisting}[language=sh]
$ build/install/mainExecutable/main.bat
1. Hello Mr. Pepito (Community) 
2. Hello Mr. Pepito (Community) 
3. Hello Mr. Pepito (Community) 
4. Hello Mr. Pepito (Community) 
5. Hello Mr. Pepito (Community) 
6. Hello Mr. Pepito (Community) 
7. Hello Mr. Pepito (Community) 
8. Hello Mr. Pepito (Community) 
9. Hello Mr. Pepito (Community) 
10. Hello Mr. Pepito (Community) 
\end{lstlisting}
\caption{build/install/mainExecutable/main.bat}
\end{algorithm}



\subsubsection{Distintos ``sabores''}

Con unas pocas líneas más, podemos generar distintas versiones de
la misma aplicación, en nuestro ejemplo vamos a generar una versión
``Community'' y otra ``Enterprise''.

\begin{algorithm}[H]
\begin{lstlisting}
apply plugin: 'cpp'
model {
	flavors {
		community
        enterprise
    }
}
libraries {
 	hello {
        binaries.all {             
			if (flavor == flavors.enterprise) {                 		cppCompiler.define "ENTERPRISE"
            }
        }
    }
}
executables {
    main {
        binaries.all {
            lib libraries.hello.shared
        }
    }
}
\end{lstlisting}
\caption{build.gradle}
\end{algorithm}


Además tenemos que preparar nuestra aplicación para utilizar estos
parámetros de compilación. 

\begin{algorithm}[H]
\begin{lstlisting}[language={C++},tabsize=4]
#ifdef ENTERPRISE
static const char * EDITION = "Enterprise";

#else 
static const char * EDITION = "Community";

#endif
\end{lstlisting}


\caption{Msg.h}
\end{algorithm}
encuentra en Todo el proyecto se 

De esta forma se utiliza una cadena u otra en función del ``sabor''
con que compilemos. 

Si ahora ejecutamos gradle clean task en la raíz de nuestro proyecto,
veremos que tenemos más tareas disponibles, antes teníamos installMainExecutable
y ahora ha sido reemplazada por installCommunityMainExecutable y
installEnterpriseMainExecutable.

Si ejecutamos estas dos tareas , tendremos nuestra aplicación instalada
en los dos sabores.

\begin{algorithm}[H]
\begin{lstlisting}[language=sh]
$gradle installEnterpriseMainExecutable installCommunityMainExecutable

:compileEnterpriseHelloSharedLibraryHelloCpp 
:linkEnterpriseHelloSharedLibrary 
:enterpriseHelloSharedLibrary 
:compileEnterpriseMainExecutableMainCpp 
:linkEnterpriseMainExecutable 
:enterpriseMainExecutable 
:installEnterpriseMainExecutable 
:compileCommunityHelloSharedLibraryHelloCpp 
:linkCommunityHelloSharedLibrary 
:communityHelloSharedLibrary 
:compileCommunityMainExecutableMainCpp 
:linkCommunityMainExecutable 
:communityMainExecutable 
:installCommunityMainExecutable

BUILD SUCCESSFUL
Total time: 9.414 secs 
\end{lstlisting}
\caption{gradle installEnterpriseMainExecutable installCommunityMainExecutable}
\end{algorithm}


Ahora podemos ejecutar nuestra aplicación en los dos sabores:

\begin{algorithm}[H]
\begin{lstlisting}[language=sh]
$ build/install/mainExecutable/community/main.bat
1.      Hello Mr. Pepito        (Community)
2.      Hello Mr. Pepito        (Community) 
3.      Hello Mr. Pepito        (Community) 
4.      Hello Mr. Pepito        (Community) 
5.      Hello Mr. Pepito        (Community) 
6.      Hello Mr. Pepito        (Community) 
7.      Hello Mr. Pepito        (Community) 
8.      Hello Mr. Pepito        (Community) 
9.      Hello Mr. Pepito        (Community) 
10.     Hello Mr. Pepito        (Community)
\end{lstlisting}
\caption{Community}
\end{algorithm}
\begin{algorithm}[H]
\begin{lstlisting}[language=sh]
$ build/install/mainExecutable/enterprise/main.bat 
1.      Hello Mr. Pepito        (Enterprise) 
2.      Hello Mr. Pepito        (Enterprise) 
3.      Hello Mr. Pepito        (Enterprise) 
4.      Hello Mr. Pepito        (Enterprise) 
5.      Hello Mr. Pepito        (Enterprise) 
6.      Hello Mr. Pepito        (Enterprise) 
7.      Hello Mr. Pepito        (Enterprise) 
8.      Hello Mr. Pepito        (Enterprise) 
9.      Hello Mr. Pepito        (Enterprise) 
10.     Hello Mr. Pepito        (Enterprise)
\end{lstlisting}
\caption{Enterprise}
\end{algorithm}



\subsubsection{Release o Debug}

Por defecto Gradle compila nuestra aplicación en modo Debug, pero
podemos añadir el modo Release para que active algunas optimizaciones%
\footnote{También podemos definir el tipo de optimizaciones que vamos a utilizar.%
}.

\begin{algorithm}[H]
\begin{lstlisting}[language=sh]
apply plugin: 'cpp'
model {
    buildTypes {
        debug         
		release
    }
// ... the rest of file below doesn't change 
\end{lstlisting}
\caption{build.gradle}
\end{algorithm}


Si ahora ejecutamos gradle clean task veremos que tenemos más tareas,
se habrán desdoblado las que teníamos, por ejemplo installCommunityMainExecutable
se habrá desdoblado en installDebugCommunityMainExecutable y installReleaseCommunityMainExecutable.


\subsubsection{Multi-plataforma}

También tenemos las posibilidad de utilizar las características de
compilación cruzada que nos ofrecen los compiladores y generar componentes
nativos para otras aplicaciones. El proceso es el mismo, simplemente
tenemos que dar te alta las aplicaciones con las que vamos a trabajar. 

Esto solo funcionará si en nuestro sistema tenemos instalada la cadena
de herramientas (\href{http://es.wikipedia.org/wiki/Cadena_de_herramientas}{Toolchains})
necesaria, es decir, si en un sistema de 64 bits queremos compilar
para 32 bits, tendremos que tener instaladas las librerías necesarias
en en 32 bits. 

\begin{algorithm}[H]
\begin{lstlisting}[language=sh]
apply plugin: 'cpp'
model {
    buildTypes {
        debug
        release
    }
         platforms {
        x86 {
            architecture "x86"
        }
        x64 {
            architecture "x86_64"
        }
        itanium {
            architecture "ia-64"
        }
    } 
    flavors {
        community
        enterprise
    }
}
libraries {
 	hello {
        binaries.all {
            if (flavor == flavors.enterprise) {
                cppCompiler.define "ENTERPRISE"
            }
        }
    }
}
executables {
    main {
        binaries.all {
            lib libraries.hello.shared
        }
    }
}
\end{lstlisting}
\caption{build.gradle}
\end{algorithm}


Ejecutando gradle clean task podremos generar distintas versiones
de nuestra aplicación en distintos sabores, para distintas aplicaciones
en Debug o Release. 


\section{Conclusiones}

Con una configuración mínima, tenemos muchas posibilidades de construcción
de aplicaciones nativas multi-plataforma. 

Tiene un futuro prometedor, veremos como termina. 

Podemos utilizar otras características de Gradle y aplicarlas a nuestros
proyectos C++, como análisis estáticos de código, generación de informes
de prueba, fácil incorporación a sistemas de integración continua. 

Gradle para C++ es una característica que actualmente está en desarrollo,
por lo que:
\begin{itemize}
\item No debemos utilizar en entornos reales de desarrollo, puede acarrear
muchos dolores de cabeza.
\item La forma de definir el fichero build.gradle puede cambiar. 
\end{itemize}
Todo el ejemplo se encuentra en \href{https://github.com/carlosvin/cpp_gradle}{https://github.com/carlosvin/cpp\_{}gradle}.
\end{document}
